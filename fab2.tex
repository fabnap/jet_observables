
\subsection{Track-Assisted Mass ($\mta$)}\label{subsec:mta}
The main limitation of the calorimeter mass comes from the angular resolution of the topo-clusters, which, for extreme kinematic regimes, start approaching each other at the point that they hit the granularity of the detector. The main advantage is that on the contrary the relative energy resolution increases at higher energies.

The tracks instead have a very good angular resolution, but $\pt$ relative resolution degrades linearly with the transverse momentum. 

One could then think about creating a variable which exploits the advantages of both and minimizes the disadvantages. As seen, the track mass is missing the neutral component, i.e. each measurement is missing the fraction $\frac{neutral+charged}{charged}$, but it could be corrected on a jet-by-jet basis: this leads to the definition of the \textit{track-assisted mass} ($m^{TA}$):
\begin{equation}
 m^{TA}=\frac{p_T^{calo}}{p_T^{track}}\times m^{track}
\end{equation}

It can be intuitively understood as follows: the term $m^{track}$ has the superior angular resolution, but misses the neutral component; the ratio $p_T^{calo}/p_T^{track}$, representing exactly the $(neutral+charged)/charged$ ratio, ``restores'' the correct value of the mass back to $charged+neutral$.
\begin{figure}[!ht]
  \centering
      \includegraphics[width=0.7\textwidth]{jet_part/mta/allbinptmta.png}
  \caption[$\mcal$ and $\mta$ mass responses]{Track-assisted mass response plot for boosted $W/Z$: in green the calorimeter mass, in red the track-assisted mass. On the right are shown properties of the fit to the Gaussian core; it can be seen than the width of the $\mta$ distribution is smaller, and the mean is slightly below the calorimeter mass.}
  \label{fig:mta1}
\end{figure}

From Figure \ref{fig:mta1} the comparison of the track-assisted mass and the calorimeter mass; the width of the distribution is smaller, making this observable a good candidate for usage.


\subsection{Advantages and Limitation of $\mta$}
The $\mta$ has a good handle on boosted $W/Z$, looking at all the transverse momentum spectrum for these results.

\begin{figure}[!ht]
  \centering
      \includegraphics[width=0.9\textwidth]{jet_part/uncert.png}
  \caption[Comparison of the uncertainties for $\mcal$ and $\mta$]{Comparison of the uncertainties for $\mcal$, on the left, and $\mta$, on the right the rise on the high jet $\pt$ is due to statistics. From the \cite{art35}.}
  \label{fig:uncert}
\end{figure}

Another big advantage which supports the use of the track-assisted mass is the relatively small uncertainties: in Figure \ref{fig:uncert} the comparison of $\mcal$ (left) and $\mta$ (right) fractional uncertainties on the JMS, shows how the tracking uncertainties are much smaller because of the ratio $m^{track}/p_T^{track}$. On the right plot the black line indicates the JMS fractional uncertainty for the $\mcal$, and is always above the $\mta$. Of course this introduces another argument in the development of new techniques, which is to look for a good balance between performance and small uncertainties: a perfect observable in terms of behavior which has very big uncertainties is not really useful.


When looking in the extreme kinematic regime, at very high $\pt$, as in the top plot in Figure \ref{fig:mta2}, the $\mta$ shows its real strength, achieving much smaller value of the IQnR.
However, there are some severe limitations which are worth noting, especially looking at the performance in different regions of transverse momentum: this is shown in the bottom plot of Figure \ref{fig:mta2}, where at a low $\pt$ it exhibits a much worse behavior.

\subsubsection{Performance in $W \to q'\bar{q}$ Decays}

\begin{figure}
    \centering
    \begin{subfigure}[b]{0.5\textwidth}
	\centering
        \includegraphics[width=\textwidth]{jet_part/mta/highptmta.png}
   
%         \label{fig:tiger}
    \end{subfigure}
    \begin{subfigure}[b]{0.5\textwidth}
	\centering
        \includegraphics[width=\textwidth]{jet_part/mta/lowptmta.png}
 
%         \label{fig:gull}
    \end{subfigure}
    \caption[Mass response plots for the $\mta$]{Mass response plots for selected ranges of $\pt$: on the bottom, a ``low'' range, 500 GeV $<\pt<$ 700 GeV, on the top an high $\pt$, 1900 GeV $<\pt<$ 2100 GeV. A difference in performance can be clearly seen.} 
    \label{fig:mta2}
\end{figure}


The performance in all the bins of $\pt$ can be studied looking at Figure \ref{fig:mta3}; these plots have as horizontal axis the transverse momentum and as vertical one the value of the $\iqr$ calculated from the correspondingly response. For $W/Z$ jets, there is a crossing point around $\pt\sim$1 TeV, which can be understood as the point in which the two sub-jet present start merging (sub-jet multiplicity shown in Figure \ref{fig:multi} in Appendix).



\subsubsection{Performance in $t\to q'\bar{q}b$ Decays}

For top quarks the situation is much different: with respect to $W/Z$ jets, in fact, there are two main disparities: on one side, the mass of the top quark is much higher than the one of the electroweak bosons, hence making the separation $\Delta R=\frac{2m}{\pt}$ bigger; on the other side, the decay is not anymore two-prong (two-sub-jet-like) but rather a three-prong  (three-sub-jet-like) decay, one from the b-jet and the other two from the $W$ decay.
$\mta$ is here never performing better than $\mcal$, as can be seen e.g. in Figure \ref{fig:mta3}, right.


\begin{figure}[!ht]
  \centering
      \includegraphics[width=\textwidth]{jet_part/mtawandtop.png}
  \caption[$m^{calo}$ and $m^{TA}$ comparison for $W/Z$ jets and top jets]{The comparison between the performance of $m^{calo}$ and $m^{TA}$ for $W/Z$ jets (left) and top jets (right); on the x-axis the transverse momentum and on the y-axes the $\iqr$ of the mass distribution, from \cite{art35}. A better observable has lower values on the y-axis. }
  \label{fig:mta3}
\end{figure}

\subsubsection{Performance in $h\to b\bar{b}$ Decays}

For boosted Higgs the $\mcal$ outperforms the $\mta$ in the spectrum of transverse momentum. Although the decay is two-pronged, the mass of the Higgs is higher than the electroweak bosons, moreover another difference lays in light quarks initiated jets and heavy quarks initiated ones, like the b-quarks from Higgs decay.
% the b-jet poses an additional complication which comes from the branching ratio of B mesons to muons, which leave very little energy in the calorimeter system but additional tracks.

\begin{figure}[!ht]
  \centering
      \includegraphics[width=0.7\textwidth]{jet_part/mta/higgsmta.png}
  \caption[Performance of the $\mta$ with the boosted Higgs sample]{Performance of the $\mta$ with the boosted Higgs sample; the $\mta$ is the blue line, the $\mcomb$ will be described later in this chapter. From \cite{art39}. The FoM here is the resolution of the Response.}
  \label{fig:mta4}
\end{figure}



\subsection{The Track-Assisted Sub-jet Mass ($\mtas$)}\label{subsec:mtas}
In this section the main outcome of the work of this thesis is presented: the \textit{track-assisted sub-jet mass} ($\mtas$).
The main idea takes inspiration from the track-assisted mass: if one can use the tracks to exploit the better angular resolution and correct the missing neutral component jet-by-jet, there is an additional information that can be used. The neutral fraction, in fact, varies stochastically not only per-jet basis, but even per-sub-jet basis, since each sub-jet is originated from a different quark.
Correcting the missed neutral component per-sub-jet, it should perform better already at an intuitive level, as it accesses information from the jet substructure.
There are few question in the definition of this mass observable, whose answers are in the next section:
\begin{itemize}
  \item Regarding the inputs:
  \begin{itemize}
     \item How to select the set of tracks to be used?
     \item Which kind of sub-jet should be used?
  \end{itemize}
  \item Regarding the procedure
  \begin{itemize}
  
  \item How to associate the tracks to a sub-jet?
  \item How to correct for the missed neutrals on a sub-jet basis?
  \item How to add everything back together?
 \end{itemize} 
 
\end{itemize}

Those details are given in the next subsection.


\subsection{Observable Definition: Inputs}
There are two inputs to the $\mtas$: the tracks and the sub-jets. The definition of the standard inputs are give here; alternative approaches are given in subsection \ref{sec:alternate}.

\subsubsection{Tracks}
Only the tracks that satisfy the quality criteria and primary vertex association, described in the previous section \ref{sec:tracks}, are used.
The tracks taken additionally are required to be ghost associated to the sub-jets of the groomed jet; namely only the sub-jets which survived the trimming procedure and are described in the next subsection.
Ghost association provides a one-to-one correspondence to the sub-jets set, and was therefore chosen and preferred to other kind of assignments.

\subsubsection{Sub-jets}

The choice of sub-jets must follow a simple requirement: of course we want to take those which most likely come from the hard-scattering. This means that the choice of taking them after grooming is forced.

As grooming technique used, the trimming was preferred as being the standard in ATLAS and the most flexible one for optimization studies.

The standard version of the trimming uses the k$_t$ reclustering algorithm with radius of 0.2, with the transverse momentum ratio $f_{cut}$ at 5\%.

As shown later, this is also the optimal configuration for sub-jets.

\subsection{Observable Definition: Procedure}\label{subsec:ObsDef_Proc}
Having tracks and sub-jets now well defined, we can describe the recipe to produce the $\mtas$. For brevity we will call the sub-jets SJ in the formulae below. 

As said, the tracks are the one ghost-associated to the sub-jets; however, tracks which fall inside the area of the large-$R$ jet, but not inside the sub-jets area, are still much probably coming from the hard-scattering. They are then associated again to the closest sub-jets via $\Delta R$ association.

Each sub-jet will have at this point some tracks associated via ghost-association and some other via $\Delta R$ (which are maximally 5\%). We call this set of tracks, a ``custom'' Track-Jet or TJ.

At this point, the one-to-one correspondence is still preserved (for each SJ there is one and only one TJ), and we can move on correcting the neutral fraction.

Getting inspired from the formula $m^{TA}=p_T^{calo}/p_T^{track}\times m^{track}$, we would like to replicate this at sub-jet level, i.e.

$$\mtas="\sum_{SJ}"\frac{p_T^{SJ}}{p_T^{TJ}}\times m^{TJ}$$

Since now we are working inside the sub-jets we need to change the sub-jet's 4-vector itself and not only the mass: if we call $p_\mu^{TJ}$ the Lorentz vector of the track-jet, 

$$p_\mu^{TJ} = \spvec{m^{TJ};p_T^{TJ};\eta^{TJ};\phi^{TJ}} \to p_\mu^{TA}=\spvec{m^{TJ}\times\frac{p_T^{SJ}}{p_T^{TJ}} ;p_T^{SJ};\eta^{TJ};\phi^{TJ}} $$
 
where $p_\mu^{TA}$ is the track-assisted sub-jet's 4-vector. If we label $i$ the $i$-th track-jet of the $N$ ones present in the large-$R$ jet,

$$ \mtas=\sqrt{\left(\sum_i^N p^{TA} \right)_\mu \left(\sum_i^N p^{TA} \right)^{_\mu}} $$
 
\begin{figure}[!ht]
  \centering
      \includegraphics[width=0.6\textwidth]{jet_part/mtas/mtas.png}
  \caption[Pictorial event display]{Pictorial event display showing the $\eta$ $\phi$ region of a large-$R$ jet, (in blue the catchment area of the anti-k$_t$) showing the different k$_t$ sub-jets: they are highlighted in green, fuchsia and yellow. The associated track-jets (here as arrows pointing the calorimeter area) are colored with the same color of the correspondent sub-jet. Some tracks associated with $\Delta R$ procedure can be seen in the fuchsia sub-jet. The transverse momenta and mass values are also shown for the sub-jets.}
  \label{fig:mtas1}
\end{figure}

An important remark is that, in the case of a large-$R$ jet with only one sub-jet, the $\mtas$ has exactly the same definition of the $\mta$. This implies, since the angular separation of the decay product scales inversely with $\pt$, that the performance should approach the one of the $\mta$ in the extreme kinematic regime. However, the space for improvement is precisely in the low-middle $\pt$ regime, as seen in the $\mta$ section.

\subsection{Performance in $W \to q'\bar{q}$ Decays}
The boosted $W/Z$ was the first one looked at, and with which the $\mtas$ was designed. The $\mcal$ shows a fast deterioration of the performance at high $\pt$, and, as shown in the previous section, the $\mta$ prevents this deterioration but suffers at low transverse momenta ($\pt<1$ TeV).
The $\mtas$ has the same behavior in the extreme transverse momentum regime as the $\mta$, since the sub-jet multiplicity peaks at one, where there are no differences between the two observables.
In the low-$\pt$ regime, on the contrary, it exploits the different charged to neutral fluctuation, achieving a better performance.
This is shown in Figure \ref{fig:mtas2} as a function of $\pt$: below $\sim$ 1 TeV ic achieves lower values of the IQnR converging from below to the $\mta$ as the number of sub-jets decreases to one.

\begin{figure}[!ht]
  \centering
      \includegraphics[width=0.7\textwidth]{jet_part/mtas/71graphcftr_h_JetRatio_mJ12CALOIQRoMWZ.pdf}
  \caption[$\mtas$ for boosted $W/Z$]{Performance of the $\mtas$ versus the $\mcal$ and $\mta$ for the boosted $W/Z$ sample.}
  \label{fig:mtas2}
\end{figure}

\subsection{Performance in $t\to q'\bar{q}b$ Decays}
The boosted tops are shown on Figure \ref{fig:mtas3}; the $\mtas$ is comparable yet slightly worse than the $\mcal$ in the low-middle $\pt$ regime, while degrades at higher $\pt$ approaching the $\mta$, which is far beyond the track-assisted sub-jet mass in performance.
As already noted, the worse performance can be ascribed both to the higher top-quark mass, and to its different and more complex decay topology.


\begin{figure}[!ht]
  \centering
      \includegraphics[width=0.7\textwidth]{jet_part/mtas/71graphcftr_h_JetRatio_mJ12CALOIQRoMTops.pdf}
  \caption[$\mtas$ for boosted tops]{Performance of the $\mtas$ versus the $\mcal$ and $\mta$ for the boosted top sample.}
  \label{fig:mtas3}
\end{figure}
\subsection{Performance in $h\to b\bar{b}$ Decays}
In the Randall-Sundrum graviton to di-Higgs to four b-quark, the performance is again problematic for the $\mta$ with respect to $\mcal$, which is far beyond the latter, while the performance of the $\mtas$ is partially similar to the boosted top-quark sample, but degrades much more in the extreme $\pt$ regime, following the $\mta$. Shown in Figure \ref{fig:mtas4}.

\begin{figure}[!ht]
  \centering
      \includegraphics[width=0.7\textwidth]{jet_part/mtas/71graphcftr_h_JetRatio_mJ12CALOIQRoMHiggs.pdf}
  \caption[$\mtas$ for boosted Higgs]{Performance of the $\mtas$ versus the $\mcal$ and $\mta$ for the boosted Higgs sample.}
  \label{fig:mtas4}
\end{figure}

\subsection{Performance in QCD Multijet Events}
The behavior of the QCD multijet sample is similar to the boosted $W/Z$ sample, where the $\mta$ exhibits a crossing point in the middle-low regime $\pt\simeq900$ GeV and proceeds with a better performance at high transverse momenta.
Again the $\mtas$ follows this similarity showing no crossing point and an optimal overall behavior, both with respect to calorimeter- and track-assisted-based mass definition. On Figure \ref{fig:mtas5}.

\begin{figure}[!ht]
  \centering
%       \includegraphics[width=\textwidth]{jet_part/mtas/qcdmtas.png}
        \includegraphics[width=0.7\textwidth]{jet_part/mtas/qcdmtastruffa.png}
   \caption[$\mtas$ for QCD jets]{Performance of the $\mtas$ versus the $\mcal$ and $\mta$ for the QCD multijet. Here shown IQR/Med not $\iqr$.}
  \label{fig:mtas5}
\end{figure}

\subsection{Performance in Massive $\tilde{W}\to q'\bar{q}$ Decays with $m_{\tilde{W}}=m_t$}
The massive $W$ sample is a special sample which was used to understand the behavior of the boosted tops, whether its worse resolution was coming from the higher mass of the top quark or from the more complex decay topology (three-pronged instead of two-pronged decay and b-quark presence). 
The sample is almost identical to the boosted $W/Z$ one ($W'\to WZ$) but in this case the SM electroweak boson are set to have the mass of the top quark $m_{\tilde{W}}=m_t$.
In fact, from the rule $\Delta R=2m/p_T$, a bigger separation is expected between the quark from the hadronic decay.
The comparison with $\mcal$ is shown in Figure \ref{fig:mtas6}, together with the boosted top-quark for comparison. As seen here, the performance of the latter is clearly worse than the former, the trend is yet very similar. This difference is interpreted in terms of different and more complex topology and hence higher sub-jet multiplicity: in the three sub-jet structure, resolving accurately the components is more challenging.

\begin{figure}[!ht]
  \centering
     \includegraphics[width=0.7\textwidth]{jet_part/mtas/71graphcftr_h_JetRatio_mJ12CALOIQRoMcalib_WmassiveVsTops.pdf}
   \caption[$\mtas$ for boosted massive $W/Z$]{Performance of the $\mtas$ versus the $\mcal$ for the massive $W/Z$ (in red and green); shown on the same plot also the boosted top sample (in blue and light blue).}
  \label{fig:mtas6}
\end{figure}

\subsection{Other Stability Quantifiers}
The stability of the $\mtas$ was checked, although the IQnR is already a good quantifier of stability, explicitly for the mean of the mass response distribution and for the left-hand-side tail, as a function of the transverse momentum. This was an important check to assure the overall gaussianity of the final distribution in the whole spectrum of $\pt$, and suitability in regards of the calibration step, which is not discussed in this thesis.

The mean of the response distribution is shown for boosted $W/Z$ decays in Figure \ref{fig:meanandtail}, left; as seen here, despite being the mean constantly below the unity, its behavior is much more flat and independent of $\pt$, especially in the low-middle regime. This is surprising since the $\mcal$ is already shown after the calibration step, which is not taken instead for the $\mtas$. Conversely the left-hand-side tail of the mass response which is shown in the same figure, right, shows a more enhanced behavior than the $\mcal$, but still never reaches the 10\%. Of course an enhancement of the tail causes a loss of gaussianity and a number of jets which are reconstructed with a lower mass than they should, but it is still comparable with the calorimeter mass.

Those quantifiers show analogous behavior for the other samples considered and those figures can be found in the Appendix.

\begin{figure}
    \centering
    \begin{subfigure}[b]{0.45\textwidth}
	\centering
        \includegraphics[width=\textwidth]{appendixB/mTAS_W_calibmCal_20:07:01-03-11-2016/71graph_h_JetRatio_mJ12CALO_meanResponseMvsTA.pdf}
%         \label{fig:tiger}
    \end{subfigure}
    \begin{subfigure}[b]{0.45\textwidth}
	\centering
        \includegraphics[width=\textwidth]{appendixB/mTAS_W_calibmCal_20:07:01-03-11-2016/74graph_h_JetRatio_mJ12CALO_I50ResponseMvsTAnorm.pdf}
 
%         \label{fig:gull}
    \end{subfigure}
    \caption[Mean and left-hand side integral for boosted $W/Z$]{Stability quantifiers which were checked for the $\mtas$: mean, on the left, and normalized left-hand side integral, on the right, of the mass response distribution. The mean is calculated from a Gaussian fit and the integral goes from 0 to 0.6.} 
    \label{fig:meanandtail}
\end{figure}

\subsection{Sub-jet Calibration}

An additional attempt of calibrating the sub-jet was also tried and, although the results were not substantially improved, it is presented in this section. This study was performed using only boosted $W/Z$ samples.

\subsection{Preliminary Studies on Sub-jet Calibration}
The first attempt in calibrating the sub-jets had as start a ``perfect calibration'', which means using the truth-level information from the MC sample \textit{before} the interaction with the calorimeter.
Truth-level tracks are the particles in the jet which have an electric charge and are stable, truth-level sub-jets are all the particles, charged and not, which are ghost associated to the calorimeter sub-jets.
There are few possibilities in doing so, here some nomenclature for this study will be introduced:
\begin{itemize}
 \item $\mtas$ using truth-level sub-jets and tracks; normal tracks (with all detector effects) are used to assist the truth-level sub-jets;
 \item $\mtas$ using truth-level tracks and truth-level sub-jets; the truth-level tracks are used to assist the truth-level sub-jets;
 \item $\mcal$ truth, calculated using only the truth sub-jets.
\end{itemize}


\subsubsection{Perfect Calibration}
The \textit{perfect calibration} refers to the procedure of using $\mtas$ with truth-level sub-jets and track, i.e. looking at the best possible scenario with an ideal detector. The performance is of course expected to be optimal, because of the use of the truth-level. This step was necessary as feasibility study, to understand whether ulterior efforts in this direction were meaningful.
The perfect calibration is shown in Figure \ref{fig:perfcalib}; since the performance exhibits room for big improvement below $\sim$ 1 TeV and moderate to small improvement above this value, the second step of a simple calibration was tried.

\begin{figure}[!ht]
  \centering
      \includegraphics[width=0.7\textwidth]{jet_part/calib/perfcalib.png}
  \caption[Perfect calibration]{Performance of the perfect calibration. It shows room for improvement especially at low-middle $\pt$.}
  \label{fig:perfcalib}
\end{figure}


\subsubsection{Simple Sub-jet Calibration}
Following the example of calibration of jets in general, a simple approach to emulate this procedure was tried, constructing in various bins of transverse momenta the responses of the sub-jet's energy to derive the weights factors to be applied. The detailed procedure is as follows:
\begin{enumerate}
 \item Responses in energy $R_E=E^{reco}/E^{truth}$ were built in several bins of $\pt$, spanning to the whole transverse momentum range;
 \item The mean $\mu_R$ of this response was calculated via a fit to the Gaussian core;
 \item Those values (\textit{scale factors}) were stored and applied again to the sub-jets before the computation of the $\mtas$ via 4-momentum correction $E'=E/\mu_R$; the $\pt$ (the value which only enters the $\mtas$ variable) was changed then correspondingly to keep the sub-jet's mass constant.
\end{enumerate}

This procedure was called \textit{poor man's calibration} or PM calibration or \textit{simple calibration}.
A check on the $\pt$ response before and after calibration together with the mean of the entire Large-$R$ jet response is shown in Figure \ref{fig:calibA} and \ref{fig:calibA2} in Appendix.

The results are on Figure \ref{fig:perfcalib4}; there are only marginal improvements in few ranges of low transverse momentum where the scale factors are further away from unity, and the overall observable is not performing better than the standard $\mtas$. This is interpreted both in terms of a missing calibration as a function of the $\eta$ variables (having hence a befit from the crack region) and because the correction done on average does not provide the sufficient handle in a jet-by-jet basis, especially when all the sub-jets are rescaled by similar factors (which translates into a similarity of $\pt$s of the sub-jets, often the case for e.g. boosted $W/Z$, less for boosted tops entirely contained in the large-$R$ jet).

\begin{figure}[!ht]
  \centering
      \includegraphics[width=0.7\textwidth]{jet_part/calib/perfectcalib4.png}
  \caption[Simple calibration]{Performance of the poor man's calibration. The improvement is marginal throughout the entire transverse momentum space.}
  \label{fig:perfcalib4}
\end{figure}

\subsection{Limitation of $\mtas$}
The final effort to understand the various and competing effects, which take place in the $\mtas$ and which was inspired by the perfect calibration procedure, brought to a final study on the variable to understand the reason for the worsening of the resolution at high transverse momenta, using again the truth MC information.

The preliminary investigation in this direction was then the study on the track-resolution: since the track relative resolution of the transverse momentum is expected to worsen linearly with this variable, a response of the mass of the tracks was constructed, using the truth-level tracks.

The result is shown on Figure \ref{fig:trackdegrade}: for the samples considered, it shows a linear degradation of the mass of the tracks, both for massive and SM $W/Z$.

\begin{figure}[!ht]
  \centering
      \includegraphics[width=0.7\textwidth]{jet_part/calib/71graphcftr_h_JetRatio_mJ12CALOIQRoMcalib_trkmass.pdf}
  \caption[Track mass degradation in tops and massive $W/Z$]{The performance of the track mass in blue and red for massive $W$ sample and boosted $W/Z$ respectively; for reference in green the calorimeter mass of the large-$R$ jet.}
  \label{fig:trackdegrade}
\end{figure}

The hypothesis of the degradation of the $\mtas$ driven by the tracks is also supported by the Figure \ref{fig:breakdown1} in Appendix, where the truth-level tracks are used instead of real tracks to compute the variable; it can be seen the flat behavior at high $\pt$, hence ascribing the worsening of the resolution to tracks at higher transverse momenta.


A complete breakdown of the variable in terms of truth-level particles is given in Figure \ref{fig:breakdown2}, where all the different components are separated.
In particular the black dots show the $\mtas$ using truth-level sub-jets but real tracks for the track assistance procedure.
Even combining this truth-level information, in fact, it shows a large worsening of the performance (truth-level sub-jets only are shown as blue dots).

% Particularly interesting is the black dots, which 
% uses truth-level sub-jet but real tracks, which worsen the overall performance of the truth-level sub-jet alone (shown in light blue dots).

\begin{figure}[!ht]
  \centering
      \includegraphics[width=0.7\textwidth]{jet_part/calib/71graphcftr_h_JetRatio_mJ12CALOIQRoM4Truths.pdf}
  \caption[Breakdown of the $\mtas$ ]{Breakdown of the $\mtas$ in its component using truth-level information for boosted $W/Z$ decays.}
  \label{fig:breakdown2}
\end{figure}

Other results using truth-level information on boosted tops are shown and described in the Appendix.


% \subsection{Large-R jet: Calibration}
% The jet mass scale calibration aims to correct the reconstructed jet mass to the particle-level jet mass by applying calibration factors derived from a sample of simulated QCD multijet events, with an analogous procedure described in \ref{sec:calib} for the jet energy scale.
% 
% \subsection{Large-R jet: Uncertainties}


\subsection{Alternative Observable Definitions}
\label{sec:alternate}

There are quite a few ways to modify the track-assisted sub-jet mass; however, all the alternative approaches showed worse performance, and they are mentioned here for completeness only.

Alternatives considered were: 
\begin{itemize}
 \item for the tracks:
 \begin{itemize}
   \item use of tracks not as input directly, but only taking those belonging to anti-k$_t$ reclustered track-jet with radius of 0.3 or 0.2;
   \item tighter or looser quality conditions were explored;
   \item tighter or looser primary vertex association requirement were explored.
 \end{itemize}
 \item for the sub-jets:
  \begin{itemize}
   \item the trimming procedure was modified: various radii $R_{sub}$ of the sub-jets were tested;
   \item the sub-jets were reclustered using not only the standard k$_t$, but also anti-k$_t$ and C/A.
  \end{itemize}
  \item for the procedure: different 4-momentum correction scheme was also explored.
\end{itemize}

The different reclustering algorithm choice has a deep impact and was studied in details, since it changes the topo-cluster added to the sub-jets and the tracks associated to them. The situation is depicted in the event-display in Figure \ref{fig:evtdspl}; the display on the left shows the standard choice of k$_t$, the one on the right shows the modified approach anti-k$_t$. 

In the Appendix, figure \ref{fig:allalgow} \ref{fig:allalgotop} \ref{fig:allalgohiggs} the performance for boosted $W/Z$, tops and Higgs are shown, respectively. It can be seen that the k$_t$ algorithm provides the best observable definition, in all the samples considered. However, the anti-k$_t$ algorithm provides similar performances; this was an important check as the jet calibration procedure currently going on in ATLAS, the \textit{R-Scan} procedures includes the anti-k$_t$ algorithm with radius of R=0.2 and aims at providing the calibration and uncertainties that could be used directly in the computation of the $\mtas$.

\begin{figure}[!ht]
  \centering
      \includegraphics[width=0.7\textwidth]{jet_part/mtas/evtdspl.png}
  \caption[Different reclustering in event display]{An example of event-display shows the differences in the reclustering algorithm used for the sub-jets: on the right  k$_t$ and on the left anti-k$_t$. Highlighted some constituents trimmed away with the second choice.}
  \label{fig:evtdspl}
\end{figure}

\section{Combining the mass observables}


Since the calorimeter large-$R$ jet mass is not explicitly used in the track-assisted (sub-jet) mass, it may be possible to improve the performance creating a new observable which combines both mass definitions.


\begin{figure}[!ht]
  \centering
      \includegraphics[width=0.7\textwidth]{jet_part/mcomb/example.pdf}
  \caption[Toy example of Gaussian combination]{A toy example of the combination of two independent Gaussian observables, in red and green, and their combination, in blue. It can be seen that the combination has a smaller width.}
  \label{fig:mcomb1}
\end{figure}

This is true for both the $\mta$ and the $\mtas$; they are introduced in the next subsections.
Provided that the two observables are nearly independent (correlation coefficient are $\sim$ 10\%, see Figure \ref{fig:mcomba1} in the Appendix), due to the Gaussian nature of the $\pt$ and mass response, the optimal combination of the two is linear\footnote{If the joint distribution of the responses is Gaussian, then one can write their probability distribution function as $f(x,y)=h(x,y)\times \exp[A(\mu)+T(x,y)\mu]$, where $x$ is the calorimeter-based jet mass response, $y$ is the track-assisted jet mass response,
$\mu$ is the common average response, and $h$, $A$,$T$ are real-valued functions. This form shows that the distribution is from the exponential family and therefore $T$ is a sufficient statistic. Since the natural parameter space is one-dimensional, $T$ is also complete. Therefore, the unique minimal variance unbiased estimator of $\mu$ is the unique unbiased function of $T(x, y) =x/\sigma^2_x \times + y/\sigma^2_y$.
See e.g. Ref. \cite{statistic} and \cite{art35} for details.}.
An example is provided in Figure \ref{fig:mcomb1}.
\subsection{Combination $\mta-\mcal$ }

For the $\mta-\mcal$ combination the observable are considered nearly independent, then
\begin{equation}\begin{split}
 \mcomb= a\times \mcal + b \times \mta,\\
 a=\frac{\sigma_{calo}^{-2}}{\sigma_{calo}^{-2}+\sigma_{TA}^{-2}} \qquad b=\frac{\sigma_{TA}^{-2}}{\sigma_{calo}^{-2}+\sigma_{TA}^{-2}}
\end{split}
\label{eq:mtacomb}
\end{equation}
where $\sigma_{calo}$ and $\sigma_{TA}$ are the $\mcal$'s and $\mta$'s resolution functions. The $\mcomb$ then is the $\mta-\mcal$ combination.

\subsection{Combination $\mtas-\mcal$ }

There is a main difference between the $\mtas$ and $\mta$ when it comes to combination: since the $\mtas$ is using sub-jet level information but $\mta$ not, the correlation with the $\mcal$ is expected to be higher.
This can be seen e.g. in the plots in Figure \ref{fig:mcomb2} (additional plots shown in Figure \ref{fig:mcomba2} in Appendix), where the correlation is not only higher for the simple $W/Z$ and Higgs jets, but above 50\% for tops. The assumption of independent variables here falls, forcing a more complete approach. The Ansatz is to take into account the correlation via the formula:
\begin{equation}
\begin{gathered}
\mcombtas= w\times\mcal + (1-w)\times\mtas,\\
w=\frac{\sigma_{TAS}^2-\rho\sigma_{calo}\sigma_{TAS}}{\sigma_{calo}^2+\sigma_{TAS}^2-2\rho\sigma_{calo}\sigma_{TAS}}
\end{gathered}
\label{eq:mtascomb}
\end{equation}
% 
where now $\mcombtas$ is the new $\mtas-\mta$ combination. This expression reduces then to the form:
\begin{equation}
\begin{gathered}
\mcombtas= a\times\mcal + b\times\mtas,\\
a=\frac{\sigma_{TAS}^2-\rho\sigma_{calo}\sigma_{TAS}}{\sigma_{calo}^2+\sigma_{TAS}^2-2\rho\sigma_{TAS}\sigma_{calo}}\qquad b=\frac{\sigma_{calo}^2-\rho\sigma_{calo}\sigma_{TAS}}{\sigma_{calo}^2+\sigma_{TAS}^2-2\rho\sigma_{TAS}\sigma_{calo}}
\end{gathered}
\end{equation}
which reduces to equation \eqref{eq:mtacomb} after simple algebra for the case when $\rho=0$. Of course, this value can be set to the value of the specific sample considered, or to an average of 0.3 if one wants to give a definition generally valid for all the cases considered; in this case, the performance would be slightly sub-optimal.

\subsubsection{Procedure}
The procedure of producing the $\mcombtas$ is defined as follows:
\begin{enumerate}
 \item For the given sample, the $\mtas$ and $\mcal$ are produced;
 \item The mass responses are also produced for the given ranges of $\pt$;
 \item For each of these responses, the value of the IQnR as defined previously is calculated and stored;
 \item The average correlation factor of 0.3 is assumed;
 \item With the formula \ref{eq:mtascomb}, $\mcombtas$ is calculated using the $\mtas$, $\mcal$ and the values stored from before.
\end{enumerate}
A remark on the procedure: the step 3. uses values of the IQnR because this was showed to be a more robust way to look at the response and fit-independent. For step 4. the correlation factor was decided to be and average of the samples considered.

Additionally, the IQnR weights are produced for each sample specifically. In order to give a sample-independent definition of the $\mcombtas$, following also the procedure adopted for the $\mcomb$, these weights could be taken from a QCD multijet sample and applied indiscriminately to the particular case. Here of course the performance would be again sub-optimal, since the variable was not developed in an ad-hoc way.

Throughout the results presented in the following sections, both observables were calculated with ad-hoc weights. Quantitative statements between them would still hold in the case of QCD weights. However, when confronting e.g. $\mtas$ with them it has to be kept in mind that in this case their performance is overestimated, since this choice, although being more general, would perform slightly worse.

\begin{figure}[!ht]
  \centering
      \includegraphics[width=0.7\textwidth]{jet_part/mcomb/mcomb2.png}
  \caption[$\mcal$ and $\mtas$ correlation plots]{The calorimeter based jet mass mass response versus the track-assisted sub-jet mass response, on the left for boosted $W/Z$ on the right for boosted tops.}
  \label{fig:mcomb2}
\end{figure}

\subsection{Performance in $W \to q'\bar{q}$ Decays}
On the boosted $W/Z$s sample, the performance of the $\mcombtas$ outperforms all the other definitions throughout all the transverse momentum space; on Figure \ref{fig:mcombtas3} they are shown for reference together with the $\mtas$. It can be noted here that the track-assisted sub-jet mass, although being sub-optimal, has comparable performance, yet presenting fewer complications due to the combination procedure.

\begin{figure}[!ht]
  \centering
      \includegraphics[width=0.7\textwidth]{jet_part/mcomb/mcombtas3.pdf}
  \caption[$\mcombtas$ on the boosted $W/Z$]{Performance of the combined mass on $W/Z$ samples; here shown the two definitions of the combined mass, $\mcomb$ and $\mcombtas$, together with the calorimeter mass and the track-assisted sub-jet mass.}
  \label{fig:mcombtas3}
\end{figure}


\subsection{Performance in $t\to q'\bar{q}b$ Decays}
The boosted top sample remains the most challenging one also with the combined mass; as seen on Figure \ref{fig:mcombtas4}, the $\mcomb$ performs quite similarly to the calorimeter based mass definition, yet behaving considerably better than the $\mtas$ especially at high transverse momentum. The $\mcombtas$, however, outperforms all the other definitions, and shows its optimal observable strength at middle $\pt$ i.e. in the range $1< \pt < 1.6$ TeV.

\begin{figure}[!ht]
  \centering
      \includegraphics[width=0.7\textwidth]{jet_part/mcomb/mcombtas4.png}
  \caption[$\mcombtas$ on the boosted tops]{Performance of the combined mass on the top sample; here shown the two definitions of the combined mass, $\mcomb$ and $\mcombtas$, together with the calorimeter mass and the track-assisted sub-jet mass.}
  \label{fig:mcombtas4}
\end{figure}

\subsection{Performance in $h\to b\bar{b}$ Decays}
Again, for the Higgs decay there are similarities as for the top sample; on Figure \ref{fig:mcombtas5} the two definitions of the combined mass, together with the simpler $\mtas$. Although this variable is lightly sub-optimal yet still comparable in the low to intermediate range in transverse momenta, where the tracks are driving a decrease in performance for the high to very-high $\pt$. The $\mcombtas$ uses this advantage to achieve optimal behavior in the entire transverse momentum spectrum, outperforming both $\mcal$ and $\mcomb$ almost everywhere.

\begin{figure}[!ht]
  \centering
      \includegraphics[width=0.7\textwidth]{jet_part/mcomb/mcombtas5.pdf}
  \caption[$\mcombtas$ on the boosted Higgs]{Performance of the combined mass on the Higgs decay; here shown the two definitions of the combined mass, $\mcomb$ and $\mcombtas$, together with the calorimeter mass and the track-assisted sub-jet mass.}
  \label{fig:mcombtas5}
\end{figure}

